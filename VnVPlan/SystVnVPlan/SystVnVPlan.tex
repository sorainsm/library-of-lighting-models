\documentclass[12pt, titlepage]{article}

\usepackage{booktabs}
\usepackage{tabularx}
\usepackage{hyperref}
\hypersetup{
    colorlinks,
    citecolor=black,
    filecolor=black,
    linkcolor=red,
    urlcolor=blue
}
\usepackage[round]{natbib}

%% Comments

\usepackage{color}

\newif\ifcomments\commentstrue

\ifcomments
\newcommand{\authornote}[3]{\textcolor{#1}{[#3 ---#2]}}
\newcommand{\todo}[1]{\textcolor{red}{[TODO: #1]}}
\else
\newcommand{\authornote}[3]{}
\newcommand{\todo}[1]{}
\fi

\newcommand{\wss}[1]{\authornote{blue}{SS}{#1}} 
\newcommand{\plt}[1]{\authornote{magenta}{TPLT}{#1}} %For explanation of the template
\newcommand{\an}[1]{\authornote{cyan}{Author}{#1}}

%% Common Parts

\newcommand{\progname}{ProgName} % PUT YOUR PROGRAM NAME HERE %Every program
                                % should have a name


\begin{document}

\title{Library of Lighting Models: System Verification and Validation Plan for 
Family of Lighting Models} 
\author{Sasha Soraine}
\date{\today}
	
\maketitle

\pagenumbering{roman}

\section{Revision History}

\begin{tabularx}{\textwidth}{p{3cm}p{2cm}X}
\toprule {\bf Date} & {\bf Version} & {\bf Notes}\\
\midrule
October 17, 2019 & 1.0 & Original Draft.\\
\bottomrule
\end{tabularx}

\newpage

\tableofcontents

\listoftables

\listoffigures

\newpage

\section{Symbols, Abbreviations and Acronyms}

\renewcommand{\arraystretch}{1.2}
\begin{tabular}{l l} 
  \toprule		
  \textbf{symbol} & \textbf{description}\\
  \midrule 
  T & Test\\
  \bottomrule
\end{tabular}\\

\wss{symbols, abbreviations or acronyms -- you can simply reference the SRS
  tables, if appropriate}

\newpage

\pagenumbering{arabic}

This document outlines a system validation and verification plan for the 
implementation of a sub-family of lighting models, based on the Commonality 
Analysis for a Family of Lighting Models. First it will cover general 
information about the system, including the particular design qualities the 
system should emphasize and any relevant documentation. Next it will outline 
the verification plans for the commonality analysis/requirements, system 
design, and implementation. It will then outline the software validation plan. 
Finally it will outline a series of representative test cases that are meant to 
test the functional and non-functional requirements, along with a traceability 
matrix mapping the test cases to particular requirements.

\section{General Information}

\subsection{Summary}
This software implements a sub-family of lighting models. The larger family and 
problem analysis is found in \ref{??}. This software aims to take in user 
specifications about the graphical scene (lights, objects, shading models, and 
an observer) and render a fully lit and shaded scene. To do this is runs 
calculations using basic optics principles to approximate light behaviour in 3D 
computer graphics.

\subsection{Objectives}

\wss{State what is intended to be accomplished.  The objective will be around
  the qualities that are most important for your project.  You might have
  something like: ``build confidence in the software correctness,''
  ``demonstrate adequate usability.'' etc.  You won't list all of the qualities,
  just those that are most important.}

\subsection{Relevant Documentation}

\begin{table}[h]
	\begin{tabular}{|p{3.5cm}|p{3cm}|p{5cm}|l|}
		\hline
	\textbf{Document Name} & \textbf{Document Type} & \textbf{Document Purpose} 
	& \textbf{Citation} \\
		\hline
		Commonality Analysis of a Family of Lighting Models& Commonality 
		Analysis & Problem domain description, and scoping to a reasonable 
		implementation size through assumptions and requirements. & \\ 
		\hline
	\end{tabular}
\end{table}
%\wss{Reference relevant documentation.  This will definitely include your SRS}

\section{Plan}
This section outlines the verification and validation plans, including any 
techniques or data sets being used in the testing process. It also outlines the 
members of the verification and validations team.
	
\subsection{Verification and Validation Team}
This section lists the members of the verification and validation team. These 
are individuals who contribute to the verification and validation of the system 
and software design. Individuals listed here have specific roles denoting the 
amount of involvement they will be having in the verification and validation 
process. Primary roles are actively working on it; secondary roles view the 
system when major submissions are made; tertiary roles are asked to contribute 
if able, but are under no obligation to participate.

The verification and validation team includes:

\begin{table}[h]
	\begin{tabular}{|l|l|p{9cm}|}
		\hline
		\textbf{Name} & \textbf{Role} & \textbf{Goal} \\
		\hline
		Sasha Soraine & Primary Reviewer& Ensure the verification and 
		validation 
		process runs smoothly.\\
		Peter Michalski & Secondary Reviewer& Ensure the logical consistency of 
		system 
		design and requirements in accordance with feedback role as expert 
		reviewer. \\
		Dr. Spencer Smith & Secondary Reviewer& Ensure reasonable coverage of 
		design 
		considerations and requirements as part of marking these documents. \\
		CAS 741 Students & Tertiary Reviewers& Ensure general consistency in 
		design and 
		requirements coverage in accordance with feedback role as secondary 
		reviewers.\\
		\hline
	\end{tabular}
\end{table}

\subsection{SRS Verification Plan}

We aim to verify the requirements listed in the Commonality Analysis in the 
following ways:

\begin{itemize}
	\item Have expert level users (familiar with graphics programming) do a 
	close read of the commonality analysis to compare it against existing 
	software tools.
	\item Review and revise requirements based on feedback from Domain Expert 
	and Secondary Reviewer of SRS.
	\item Ask Dr. Smith to review the scope to consider whether the 
	implementation scoping and thus listed requirements is inappropriate.
\end{itemize}

\subsection{Design Verification Plan}

We will be using the following methods to test the design:

\begin{itemize}
	\item Black box testing,
	\item Stress testing.
\end{itemize}

\subsection{Implementation Verification Plan}
We will be using the following methods to test the implementation:

\begin{itemize}
	\item White box testing,
	\item Rubber duck testing,
\end{itemize}


%\wss{You should at least point to the tests listed in this document and the 
%unit
%  testing plan.}
%
%\wss{In this section you would also give any details of any plans for static 
%verification of
%  the implementation.  Potential techniques include code walkthroughs, code
%  inspection, static analyzers, etc.}

\subsection{Software Validation Plan}

\wss{If there is any external data that can be used for validation, you should
  point to it here.  If there are no plans for validation, you should state that
  here.}

\section{System Test Description}
	
\subsection{Tests for Functional Requirements}



\wss{Subsets of the tests may be in related, so this section is divided into
  different areas.  If there are no identifiable subsets for the tests, this
  level of document structure can be removed.}

\wss{Include a blurb here to explain why the subsections below
  cover the requirements.  References to the SRS would be good.}

\subsubsection{Area of Testing1}

\wss{It would be nice to have a blurb here to explain why the subsections below
  cover the requirements.  References to the SRS would be good.  If a section
  covers tests for input constraints, you should reference the data constraints
  table in the SRS.}
		
\paragraph{Title for Test}

\begin{enumerate}

\item{test-id1\\}

Control: Manual versus Automatic
					
Initial State: 
					
Input: 
					
Output: \wss{The expected result for the given inputs}

Test Case Derivation: \wss{Justify the expected value given in the Output field}
					
How test will be performed: 
					
\item{test-id2\\}

Control: Manual versus Automatic
					
Initial State: 
					
Input: 
					
Output: \wss{The expected result for the given inputs}

Test Case Derivation: \wss{Justify the expected value given in the Output field}

How test will be performed: 

\end{enumerate}

\subsubsection{Area of Testing2}

...

\subsection{Tests for Nonfunctional Requirements}

\wss{The nonfunctional requirements for accuracy will likely just reference the
  appropriate functional tests from above.  The test cases should mention
  reporting the relative error for these tests.}

\wss{Tests related to usability could include conducting a usability test and
  survey.}

\subsubsection{Area of Testing1}
		
\paragraph{Title for Test}

\begin{enumerate}

\item{test-id1\\}

Type: 
					
Initial State: 
					
Input/Condition: 
					
Output/Result: 
					
How test will be performed: 
					
\item{test-id2\\}

Type: Functional, Dynamic, Manual, Static etc.
					
Initial State: 
					
Input: 
					
Output: 
					
How test will be performed: 

\end{enumerate}

\subsubsection{Area of Testing2}

...

\subsection{Traceability Between Test Cases and Requirements}

\wss{Provide a table that shows which test cases are supporting which
  requirements.}
				
\bibliographystyle{plainnat}

\bibliography{SRS}

\newpage

\section{Appendix}

This is where you can place additional information.

\subsection{Symbolic Parameters}

The definition of the test cases will call for SYMBOLIC\_CONSTANTS.
Their values are defined in this section for easy maintenance.

\subsection{Usability Survey Questions?}

\wss{This is a section that would be appropriate for some projects.}

\end{document}