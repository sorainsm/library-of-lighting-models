\documentclass{article}

\usepackage{tabularx}
\usepackage{booktabs}

\title{Reflection Report on Lights, Camera, Models!}

\author{Sasha Soraine}

\date{}

\input{../Comments}

\begin{document}

\begin{table}[hp]
\caption{Revision History} \label{TblRevisionHistory}
\begin{tabularx}{\textwidth}{llX}
\toprule
\textbf{Date} & \textbf{Developer(s)} & \textbf{Change}\\
\midrule
December 18, 2019 & Sasha Soraine & Original\\
\bottomrule
\end{tabularx}
\end{table}

\newpage

\maketitle

\section{Project Overview}

%\plt{Summarize the original project goals and requirements}
The original intent of this project was to understand lighting models well 
enough to be able to generate them on the fly when the user gives a description 
of a scene and a named lighting model/shader set. The requirements for this 
were to have the system act as a plugin for the Unity editor and parse Unity 
scenes as JSON files to manipulate and render.

It quickly became apparent in the second half of the semester that this wasn't 
going to happen the way I originally envisioned it. I settled on creating a 
Unity executable with easy to read Shader/Lighting files. While the original 
intention was to create something that showed off how well I knew lighting 
models, I think that this approach helped to solidify my knowledge of 
implementation elements in a way I wasn't expecting.

\section{Key Accomplishments}
Overall I'm quite proud of the coding that I've done. I'm not a strong coder 
and took on a project that I wanted to learn how to do (coding custom Shaders 
and Lighting Models) even though I was hesitant about my ability to learn both 
the details of the theory behind it and the new languages for implementation. 
While what I developed is exceptionally simple, it took me a long time and a 
lot of reading, Googling, and Youtube watching to figure out how to tackle the 
problem. A big stumbling block was trying to figure out how to turn off Unity's 
built-in light system so I could create my own. It took me 2 whole days of 
reading to realise that in the context of Unity Shader files are both the 
lighting model and shader code. That revelation was the only way I was able to 
create a working part of this project. I'm glad that I have something 
functional that can be played around with on the web, and I think that if I 
want to expand this work I could very easily create new components for it now 
that I know what I'm doing.
%\plt{What went well?  This can be what went well with the documentation, the
%  coding, the project management, etc.}

\section{Key Problem Areas}

Generally the thing that I think went wrong was trying to learn the topic as I 
went. I was unfamiliar with writing my own Shaders and Lighting Models, as well 
as with the details of the math for the topic. As such I flip-flopped between 
implementation details like what to implement it in (OpenGL or Unity), what it 
should be (a library or an executable), and how best to present it. I think 
that I spent a lot of time thinking about details and having deadlines slip 
away from me as I was caught trying to understand things. While I did suffer 
from issues of time management and coding problems, they would have been more 
manageable if I had a better grasp on my project and its vision from the get go.

%\plt{What went wrong?  This can be what went wrong with the documentation, the
%  technology, the coding, time management, etc.}

\section{What Would you Do Differently Next Time}
I think next time I would stick with an SRS from the start. Scoping instantly 
to one executable would eliminate a lot of the variables that I spent time 
thinking about. I also think I would pick something that I understood better; I 
spent a lot of time wrapping my head around Meshes, NormalMaps, and figuring 
out the way things are currently done that I got lost trying to design my own. 
I think a simpler project where I could have done more on the documentation 
because I wasn't worried about the implementation would have been a better idea.

\end{document}