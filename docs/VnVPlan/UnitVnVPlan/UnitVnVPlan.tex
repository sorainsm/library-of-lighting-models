\documentclass[12pt, titlepage]{article}

\usepackage{booktabs}
\usepackage{tabularx}
\usepackage{xr}
\usepackage{hyperref}
\hypersetup{
    colorlinks,
    citecolor=black,
    filecolor=black,
    linkcolor=red,
    urlcolor=blue
}
\usepackage[round]{natbib}
\usepackage{multirow}

%% Comments

\usepackage{color}

\newif\ifcomments\commentstrue

\ifcomments
\newcommand{\authornote}[3]{\textcolor{#1}{[#3 ---#2]}}
\newcommand{\todo}[1]{\textcolor{red}{[TODO: #1]}}
\else
\newcommand{\authornote}[3]{}
\newcommand{\todo}[1]{}
\fi

\newcommand{\wss}[1]{\authornote{blue}{SS}{#1}} 
\newcommand{\plt}[1]{\authornote{magenta}{TPLT}{#1}} %For explanation of the template
\newcommand{\an}[1]{\authornote{cyan}{Author}{#1}}

%% Common Parts

\newcommand{\progname}{ProgName} % PUT YOUR PROGRAM NAME HERE %Every program
                                % should have a name


\externaldocument[MG-]{../../../Design/MG/MG}
\externaldocument[MIS-]{../../Design/MIS/MIS}

\begin{document}

\title{Project Title: Unit Verification and Validation Plan for \progname{}} 
\author{Author Name}
\date{\today}
	
\maketitle

\pagenumbering{roman}

\section{Revision History}

\begin{tabularx}{\textwidth}{p{3cm}p{2cm}X}
\toprule {\bf Date} & {\bf Version} & {\bf Notes}\\
\midrule
Date 1 & 1.0 & Notes\\
Date 2 & 1.1 & Notes\\
\bottomrule
\end{tabularx}

~\newpage

\tableofcontents

\listoftables

\wss{Do not include if not relevant}

\listoffigures

\wss{Do not include if not relevant}

\newpage

\section{Symbols, Abbreviations and Acronyms}

\renewcommand{\arraystretch}{1.2}
\begin{tabular}{l l} 
  \toprule		
  \textbf{symbol} & \textbf{description}\\
  \midrule 
  T & Test\\
  \bottomrule
\end{tabular}\\

\wss{symbols, abbreviations or acronyms -- you can reference the SRS, MG or MIS
  tables if needed}

\newpage

\pagenumbering{arabic}

This document outlines the unit testing plan for \progname. The purpose of this 
is to identify the test cases that address the functionality of the modules for 
this program. This documentation alongside the System VnV Test Plan comprise 
the testing of the entire \progname system.

In this document I will first explain outline some general information about 
the document, including the software that is being tested and the scope of this 
document (i.e. the modules that this covers). This will be followed by an 
outline of the testing procedure including the testing team, and the automated 
testing tools and methods used. I then cover the unit test descriptions, 
rationale, and relationship to the MIS.

%\wss{provide an introductory blurb and roadmap of the
%  unit V\&V plan}

\section{General Information}

\subsection{Purpose}
The purpose of this document is to outline test cases for the modules of the 
\progname system. The \progname system implements several shading and lighting 
models and renders it in the Unity 3D environment. More information about this 
software can be found in its associated documentation at : [insert github url].
%\wss{Identify software that is being unit tested (verified).}

\subsection{Scope}
The scope of this document covers the modules outlined in the MIS. 

Of the modules covered, the following will not be implemented by me for this 
project:

\begin{itemize}
	\item Rendering Module \ref{MG-mUnity}
	\item JSON module \ref{MG-mJSON}
	\item Hardware Hiding module \ref{MG-mHH}
	\item Point3D module \ref{MG-mPoints}
	\item Vector module \ref{MG-mVectors}
	\item Vector Math module \ref{MG-mVecMath}	
	\item Colour module
	
\end{itemize}

These modules are not going to be implemented by me for this work as they exist 
integrated into the larger Unity system that this software is being created in. 
It's not anticipated that this development/system environment will change 
between versions of this software, so it makes sense to use the built-in 
services for these and use Unity's existing API for them. As these modules' 
services exist as part of the Unity development environment I will not be 
testing their functioning. I assume that their behaviour is correct as shipped 
with Unity.

As I am implementing all of the other modules they will need to be tested. I 
have prioritized their testing in the following order:

\begin{enumerate}

	\item 
\end{enumerate}

Several of the modules that I am choosing to implement also exist natively in 
the Unity environment. The reasoning for choosing to implement them instead of 
using the built-in modules are to two-fold; firstly, I do this to improve the 
information hiding element of the system design. By putting personal wrappers 
around the built-in functionality it future proofs the system against changes 
in the Unity API, as maintenance will only rely on changing API calls in the 
wrappers and not throughout the program. Secondly, it draws a specific parallel 
between the designed documents and the actual implemented system. This way 
designers reading the documentation can easily see where specific functionality 
is coming from and track the design decisions made in these documents all the 
way to the code.
%\wss{What modules are outside of the scope.  If there are modules that are
%  developed by someone else, then you would say here if you aren't planning on
%  verifying them.  There may also be modules that are part of your software, 
%but
%  have a lower priority for verification than others.  If this is the case,
%  explain your rationale for the ranking of module importance.}

\section{Plan}
This section outlines the verification and validation plans, including any 
techniques or data sets being used in the testing process. It also outlines the 
members of the verification and validation team.

\subsection{Verification and Validation Team}
This section lists the members of the verification and validation team. These 
are individuals who contribute to the verification and validation of the system 
and software design. Individuals listed here have specific roles denoting the 
amount of involvement they will be having in the verification and validation 
process. Primary roles are actively working on it; secondary roles view the 
system when major submissions are made; tertiary roles are asked to contribute 
if able, but are under no obligation to participate.

The verification and validation team includes:

\begin{table}[h]
	\begin{tabular}{|l|l|p{8cm}|}
		\hline
		\textbf{Name} & \textbf{Role} & \textbf{Goal} \\
		\hline
		Sasha Soraine & Primary Reviewer& Ensure the verification and 
		validation process runs smoothly.\\
		Dr. Spencer Smith & Secondary Reviewer& Ensure reasonable coverage of 
		design considerations and requirements as part of marking these 
		documents. \\
		\hline
	\end{tabular}
\end{table}

\subsection{Automated Testing and Verification Tools}\label{testingTools}
Unit testing for this system will use the Unity Test Framework (UTF) 
\url{https://docs.unity3d.com/Packages/com.unity.test-framework@1.1/manual/index.html?_ga=2.163129468.110989482.1575244303-869222737.1569978243}
This is the built in automated unit and integration testing suite for Unity. 
This was chosen for the simplicity of use in this environment; there are some 
limitations to this framework as outlined in their documentation. The results 
of the unit and integration tests are visually displayed in Unity when tests 
are run; passed tests have a green check next to their name, and failed tests 
have a red cross.

While UTF can check non-behaviour based functions, this code's functionality is 
entirely behaviour based - meaning that it relies on input from the user to 
generate a different render of the scene. This makes using the native UTF 
impossible as it does not support MonoBehaviour testing.

As the recommended programming environment for Unity is Visual Studio, I will 
leverage the tools shipped with this IDE to perform code coverage analysis to 
ensure that there is no dead code in the system. The purpose of checking for 
dead code in this system is to make sure that all the planned functionality is 
actually possible.

%\wss{What tools are you using for automated testing.  Likely a unit testing
%  framework and maybe a profiling tool, like ValGrind.  Other possible tools
%  include a static analyzer, make, continuous integration tools, test coverage
%  tools, etc.  Explain your plans for summarizing code coverage metrics.
%  Linters are another important class of tools.  For the programming language
%  you select, you should look at the available linters.  There may also be 
%tools
%  that verify that coding standards have been respected, like flake9 for
%  Python.}

\subsection{Non-Testing Based Verification}
There will be no non-testing based verification for this project. This is 
because the automatic testing should discover the code-based functionality 
issues. At that point behavioural issues with the software system are likely to 
be a fault in the implementation design. These types of issues will be caught 
in the System Tests, and so those details are outlined in the System VnV Plan 
\ref{label}.

%\wss{List any approaches like code inspection, code walkthrough, symbolic
%  execution etc.  Enter not applicable if you do not plan on any non-testing
%  based verification.}

\section{Unit Test Description}
These unit tests are designed around modules listed in the MIS (\ref{MIS-mis}).

%\wss{Reference your MIS and explain your overall philosophy for test case
%  selection.}

\subsection{Tests for Functional Requirements}
The following sections describe the unit tests that will be run for each module 
in this system. Many of these tests will deal with providing the same function 
several values that test its response/behaviour to ``bad'' data. To increase 
the readability of this document, the tests for each function will only be 
documented once, with the various values passed to it summarized in tabular 
form.
%\wss{Most of the verification will be through automated unit testing.  If
%  appropriate specific modules can be verified by a non-testing based
%  technique.  That can also be documented in this section.}

\subsubsection{Scene Module}
The following section outlines tests for the Scene module. This module looks to 
test loading the scene appropriately.
%\wss{Include a blurb here to explain why the subsections below cover the 
%module.
%  References to the MIS would be good.  You will want tests from a black box
%  perspective and from a white box perspective.  Explain to the reader how the
%  tests were selected.}

\begin{enumerate}

\item{CreatePoint\\}

Type: Functional Automatic
%\wss{Functional, Dynamic, Manual, Automatic, Static etc. Most will
%  be automatic}
					
Initial State: --
					
Input: 
\begin{tabular}{|c|c|}
	\hline
	\textbf{Variation Name} & \textbf{Passed Values}\\
	\hline
	Valid Values & x:= 10, y:= 10, z:= 10 \\
	Invalid Values & x:= A, y:= B, z:= C \\
	Empty Values & x:= null, y:= null, z:= null\\ 
	\hline
\end{tabular}
					
Output: 
%\wss{The expected result for the given inputs}
\begin{tabular}{|c|c|}
	\hline
	\textbf{Variation Name} & \textbf{Expected Output}\\
	\hline
	Valid Values & -- \\
	Invalid Values & Exception:= INVALID\_POINT\\
	Empty Values & Exception:= INVALID\_EMPTY\_POINT\\
	\hline
\end{tabular}

Test Case Derivation: The purpose of this function is to create a 3D cartesian 
point in line with the Point() access program. At this point there is no error 
checking in the system for the placement of the point relative to anything 
else, so all exceptions and tests in its creation only check that the input 
matches the function description. There is no test variation for duplication of 
points as that's currently allowed in the system.
%\wss{Justify the expected value given in the Output field}

How test will be performed: The test will be handled automatically by test 
script through the Unity Testing Framework. The tests will be coded as a C\# 
file that will be read by the TestRunner API.
					
\item{FindDistance\\}

Type: Functional Automatic
					
Initial State: Point3D has been successfully created with the following 
position P = Point(1,1,1).
					
Input: 
\begin{tabular}{|c|c|}
	\hline
	\textbf{Variation Name} & \textbf{Passed Values}\\
	\hline
	P == Q & Q := new Point(x:= 1, y:= 1, z:= 1) \\
	P is far from Q & Q := new Point(x:= 100, y:= 100, z:= 100) \\
	P is close to Q & Q := new Point(x:= -0.1, y:= 0.1, z:= -0.1) \\	
	\hline
\end{tabular}
					
Output:
\begin{tabular}{|c|c|}
	\hline
	\textbf{Variation Name} & \textbf{Expected Output}\\
	\hline
	P == Q & out:= 0 \\
	P is far from Q & out:= 171.4730299493189 \\
	P is close to Q & out:= 1.905255888325765 \\		
	\hline
\end{tabular}

Test Case Derivation: The purpose of this test case is to test the accuracy of 
this calculation. The variations in the context of two points in space can be 
categorized as either they are the same point (same position), they are close 
to each other, and they are far away from each other. For the last two 
variations (close and far) boundary test cases would be the particular integer 
or floating point bounds to test for underflow and overflow.

We don't consider the test case that a bad point Q is 
passed to this function because the bad value cases will be caught in the 
constructor tests.

How test will be performed: The test will be handled automatically by test 
script through the Unity Testing Framework. The tests will be coded as a C$\#$
file that will be read by the TestRunner API.  
\end{enumerate}

\subsubsection{Vector Module}
The following section outlines tests for the Vector module. As this module is 
an abstract data type with very basic functionality the tests for this section 
will be simple. Getter function tests will not be documented here as they will 
be assumed correct in their functionality.
%\wss{Include a blurb here to explain why the subsections below cover the 
%module.
%  References to the MIS would be good.  You will want tests from a black box
%  perspective and from a white box perspective.  Explain to the reader how the
%  tests were selected.}

\begin{enumerate}
	
	\item{CreateVector-Points\\}
	The following test case outlines the creation of a vector from two points, 
	P and Q, which represent the start and end of that vector.
	
	Type: Functional Automatic
	%\wss{Functional, Dynamic, Manual, Automatic, Static etc. Most will
	%  be automatic}
	
	Initial State: --
	
	Input: 
	\begin{tabular}{|c|c|}
		\hline
		\textbf{Variation Name} & \textbf{Passed Values}\\
		\hline
		P == Q & P:= Point(1,1,1), Q:= Point(1,1,1)\\
		Valid Values & P:= Point(1,1,1), Q:= (0,1,-1)\\
		\hline
	\end{tabular}
	
	Output: 
	%\wss{The expected result for the given inputs}
	\begin{tabular}{|c|c|}
		\hline
		\textbf{Variation Name} & \textbf{Expected Output}\\
		\hline

		\hline
	\end{tabular}
	
	Test Case Derivation: 
	
	How test will be performed: The test will be handled automatically by test 
	script through the Unity Testing Framework. The tests will be coded as a 
	C\# file that will be read by the TestRunner API.
	
	\item{CreateVector-DM\\}
	
	Type: Functional Automatic
	
	Initial State: 
	Input: 
	\begin{tabular}{|c|c|}
		\hline
		\textbf{Variation Name} & \textbf{Passed Values}\\
		\hline

		\hline
	\end{tabular}
	
	Output:
	\begin{tabular}{|c|c|}
		\hline
		\textbf{Variation Name} & \textbf{Expected Output}\\
		\hline
	
		\hline
	\end{tabular}
	
	Test Case Derivation: 
	
	How test will be performed: The test will be handled automatically by test 
	script through the Unity Testing Framework. The tests will be coded as a 
	C$\#$ file that will be read by the TestRunner API.  
	
\end{enumerate}

\subsubsection{Vector Math Module}
The following section outlines tests for the Vector Math module. As this module 
is an library of calculations, the functionality is very explicitly defined. 
For these test cases the results are checked against the mathematical results 
of the answer.
%\wss{Include a blurb here to explain why the subsections below cover the 
%module.
%  References to the MIS would be good.  You will want tests from a black box
%  perspective and from a white box perspective.  Explain to the reader how the
%  tests were selected.}

\begin{enumerate}
	
	\item{Addition\\}
	The following test case outlines the addition of two Vectors. The output of 
	all of these variations will be a new Vector that represents the addition 
	of the two input Vectors. For these cases the creation of the new Vector is 
	not checked as it is assumed to be covered by the testing of the Vector 
	module.
	
	Type: Functional Automatic
	%\wss{Functional, Dynamic, Manual, Automatic, Static etc. Most will
	%  be automatic}
	
	Initial State: -- \\
	
	Input: 
	\begin{tabular}{|c|p{8cm}|}
		\hline
		\textbf{Variation Name} & \textbf{Passed Values}\\
		\hline
		V1 == -V2 & V1:= Vector(Point(0,0,0),Point(1,1,1)), V2:= 
		Vector(Point(1,1,1),Point(0,0,0))\\
		V1 == V2 & V1:= Vector(Point(0,0,0),Point(1,1,1)), V2:= 
		Vector(Point(0,0,0),Point(1,1,1))\\
		V1 $\neq$ V2 & V1:= Vector(Point(1,2,4),Point(1,1,1)), V2:= 
		Vector(Point(10,5,0),Point(1,3,2))\\
		\hline
	\end{tabular}
	
	Output: 
	%\wss{The expected result for the given inputs}
	\begin{tabular}{|c|c|}
		\hline
		\textbf{Variation Name} & \textbf{Expected Output}\\
		\hline
		V1 == -V2 & out:= Vector(0,0,0,m:=0)\\
		V1 == V2 & out:= Vector(2,2,2,m:=3.464101615137755‬)\\
		V1 $\neq$ V2 & out:= Vector(ux,uy,uz,m)\\		
		\hline
	\end{tabular}
	
	Test Case Derivation: 
	
	How test will be performed: The test will be handled automatically by test 
	script through the Unity Testing Framework. The tests will be coded as a 
	C\# file that will be read by the TestRunner API.
	
	\item{ScalarMultiply\\}
	
	Type: Functional Automatic
	
	Initial State: -- \\
	
	Input: 
	\begin{tabular}{|c|c|}
		\hline
		\textbf{Variation Name} & \textbf{Passed Values}\\
		Valid Values & V:= Vector(1,1,1,m:=1.7320508), k:=3 \\
		Valid Values Negative & V:= Vector(1,1,1,m:=1.7320508), k:=-3 \\
		Valid Values Zero & V:= Vector(1,1,1,m:=1.7320508), k:=0\\				
		\hline
		
		\hline
	\end{tabular}
	
	Output:
	\begin{tabular}{|c|c|}
		\hline
		\textbf{Variation Name} & \textbf{Expected Output}\\
		\hline
		Valid Values & $V_{out}$:= Vector(3,3,3,m:=5.1961524)\\		
		Valid Values Negative & $V_{out}$:= Vector(-3,-3,-3,m:=5.1961524)\\	
		Valid Values Zero & $V_{out}$:= 
		Vector(0,0,0,m:=0)\\						
		\hline
	\end{tabular}
	
	Test Case Derivation: 
	
	How test will be performed: The test will be handled automatically by test 
	script through the Unity Testing Framework. The tests will be coded as a 
	C$\#$ file that will be read by the TestRunner API.  
	
\end{enumerate}

\subsubsection{Scene Module}
The following section outlines tests for the Scene module. This module is an 
abstract data type that acts as a control module for the rest of the system. 
Getters for this module are not tested/documented here. The following test 
cases rely on other modules having been tested and working correctly; as such 
this module will not check for valid construction of other objects.
%\wss{Include a blurb here to explain why the subsections below cover the 
%module.
%  References to the MIS would be good.  You will want tests from a black box
%  perspective and from a white box perspective.  Explain to the reader how the
%  tests were selected.}

\begin{enumerate}
	
	\item{CreateScene\\}
	The following test case outlines the creation of a Scene. In the system the 
	initScene program is populated with information from the InputParameters 
	module.
	
	Type: Functional Automatic
	%\wss{Functional, Dynamic, Manual, Automatic, Static etc. Most will
	%  be automatic}
	
	Initial State: --
	
	Input: 
	\begin{tabular}{|c|p{8cm}|}
		\hline
		\textbf{Variation Name} & \textbf{Passed Values}\\
		\hline
		\multirow{7}{*}{Valid Values} & height:= \\
		& width:= \\
		& depth:= \\
		& obs:= Observer() \\
		& ls:= LightSource() \\
		& os:= Object()\\
		& lightModel:= DIFFUSE \\
		\hline
	\end{tabular}
	
	Output: 
	%\wss{The expected result for the given inputs}
	\begin{tabular}{|c|c|}
		\hline
		\textbf{Variation Name} & \textbf{Expected Output}\\
		\hline
		
		\hline
	\end{tabular}
	
	Test Case Derivation: 
	
	How test will be performed: The test will be handled automatically by test 
	script through the Unity Testing Framework. The tests will be coded as a 
	C\# file that will be read by the TestRunner API.
	
	\item{ChangeScene\\}
	The following test cases outline changes made to the Scene. This captures 
	the functionality of several access programs but they all result in Scene 
	transitions.
	Type: Functional Automatic
	
	Initial State: --
	Input: 
	\begin{tabular}{|c|c|}
		\hline
		\textbf{Variation Name} & \textbf{Passed Values}\\
		\hline
		
		\hline
	\end{tabular}
	
	Output:
	\begin{tabular}{|c|c|}
		\hline
		\textbf{Variation Name} & \textbf{Expected Output}\\
		\hline
		
		\hline
	\end{tabular}
	
	Test Case Derivation: 
	
	How test will be performed: The test will be handled automatically by test 
	script through the Unity Testing Framework. The tests will be coded as a 
	C$\#$ file that will be read by the TestRunner API.  
	
\end{enumerate}

\subsection{Tests for Nonfunctional Requirements}

\wss{If there is a module that needs to be independently assessed for
  performance, those test cases can go here.  In some projects, planning for
  nonfunctional tests of units will not be that relevant.}

\wss{These tests may involve collecting performance data from previously
  mentioned functional tests.}

\subsubsection{Module ?}
		
\begin{enumerate}

\item{test-id1\\}

Type: \wss{Functional, Dynamic, Manual, Automatic, Static etc. Most will
  be automatic}
					
Initial State: 
					
Input/Condition: 
					
Output/Result: 
					
How test will be performed: 
					
\item{test-id2\\}

Type: Functional, Dynamic, Manual, Static etc.
					
Initial State: 
					
Input: 
					
Output: 
					
How test will be performed: 

\end{enumerate}

\subsubsection{Module ?}

...

\subsection{Traceability Between Test Cases and Modules}

\wss{Provide evidence that all of the modules have been considered.}

\bibliographystyle{plainnat}

\bibliography{SRS}

\newpage

\section{Appendix}

\wss{This is where you can place additional information, as appropriate}

\subsection{Symbolic Parameters}

\wss{The definition of the test cases may call for SYMBOLIC\_CONSTANTS.
Their values are defined in this section for easy maintenance.}

\end{document}