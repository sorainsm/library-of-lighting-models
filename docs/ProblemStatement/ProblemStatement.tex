\documentclass{article}

\usepackage{tabularx}
\usepackage{booktabs}

\title{CAS 741: Problem Statement\\Library of Lighting Models}

\author{Soraine, Sasha (sorainsm)}

\date{\today}

%% Comments

\usepackage{color}

\newif\ifcomments\commentstrue

\ifcomments
\newcommand{\authornote}[3]{\textcolor{#1}{[#3 ---#2]}}
\newcommand{\todo}[1]{\textcolor{red}{[TODO: #1]}}
\else
\newcommand{\authornote}[3]{}
\newcommand{\todo}[1]{}
\fi

\newcommand{\wss}[1]{\authornote{blue}{SS}{#1}} 
\newcommand{\plt}[1]{\authornote{magenta}{TPLT}{#1}} %For explanation of the template
\newcommand{\an}[1]{\authornote{cyan}{Author}{#1}}


\begin{document}

\maketitle

\begin{table}[hp]
\caption{Revision History} \label{TblRevisionHistory}
\begin{tabularx}{\textwidth}{llX}
\toprule
\textbf{Date} & \textbf{Developer(s)} & \textbf{Change}\\
\midrule
September 17, 2019 & Sasha Soraine & Initial draft.\\

\bottomrule
\end{tabularx}
\end{table}

Geometrical optics (the study of light as rays) is a well understood physical 
domain that explains how light interacts with objects (through reflection and 
refraction). This phenomena is important to understand as it is the 
mechanism that allows us (humans) to visually interact with the world around 
us. 

Understanding how to capture geometrical optics principles to 
emulate realistic lighting in an efficient manner has become a concern for many 
who deal with computer graphics in the domains of interactive digital media 
(video games), visual media (movies), and simulations. To efficiently model 
these problems requires abstracting these principles.  Geometrical optics is 
already an abstraction used to understand how light travels; capturing this 
problem for efficient use in computer graphics requires further abstraction 
without losing the essence of the problem.

In this project, I aim to understand the phenomena of light interaction with a 
three-dimensional object through the principles of reflection and refraction. 
To study this, I propose looking at existing lighting models for computer 
graphics to understand the current types of abstractions made for this problem. 
I look to separate the common elements which will relate directly to the 
physical domain, from implementation specific decisions. By capturing the 
commonalities and differences between these implementations in documentation I 
look to lay the groundwork for developing a family of lighting models to be 
implemented as a library.

%Put your problem statement here.  Comments to you can be added, like this:
%
%\wss{comment}
%
%You can also leave comments for yourself, like this:
%
%\an{comment}

\end{document}